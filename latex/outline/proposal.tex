% ============================================================
\documentclass{article}

\usepackage[margin=0.7in]{geometry}
\usepackage{cite}

% ============================================================
% custom pckgs
\usepackage{"../sty/sty_base"}
% ------------------------------------------------------------
% \input{"../sty/sty_sci"}
% \input{"../sty/sty_stat"}
% \input{"../sty/sty_linalg"}
% ============================================================

\newcommand{\needcite}{$^{[citation \ needed]}$}

\title{TITLE}
\author{Michael Ivanitskiy, Aakash Patel, Connor Puritz}
\date{\today}

\title{Math 568 project: proposal}
	
\begin{document}
\maketitle\noindent

Recently, the connectome of both sexes of \emph{C. elegans} has been mapped throughout its development\cite{connectome-development}\cite{sexes-whole-connectome}\cite{sexual-dimorphism}. The pathways governing synaptogenesis throughout development are always well studied in the organism \cite{axon-guidance-genetics}. These combined make \emph{C. elegans}, with only 302 neurons, an excellent organism for \emph{in silico} studies of synaptogenesis.

For our project, we aim to build a network-scale model of synaptogenesis with a focus on replicating chemotaxic behavior. Such a model is important to understand both decision-making at the primitive level that it occurs in \emph{C. elegans}\cite{dynamic-encoding-chemotaxis}\cite{decision-connectome}, as well as as a precursor to features like long-term memory in more complex organisms. To build such a model, we will initially simply hard-code many features, notably growth cone initiation and axon guidance molecule emission (as a function of neural activity) based on data from studies of \emph{C. elegans}.

To test our model, we will start with a \emph{C. elegans} connectome from early in development, and try to mimic synaptic growth by releasing the appropriate amounts of neurotrophins by the correct neurons. We will also simulate the electrical activity in the connectome throughout the growth period for a variety of stimuli, as these can play a role in synaptogenesis. After some time, we will compare the structure and function of the resulting network to the actual connectome of adult \emph{C. elegans}.

In our project, we aim to determine:
\begin{itemize}
	\item can synaptogenesis in \emph{C. elegans} be effectively modelled \emph{in silico}?
	\item to what extent do external stimuli play a role in connectome development of \emph{in silico} \emph{C. elegans}, and how well do these results match those done in vivo?
\end{itemize}

\bibliography{../refs}{}
\bibliographystyle{plain}

\end{document}

